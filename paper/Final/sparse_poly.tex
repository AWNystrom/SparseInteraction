\documentclass{article} % For LaTeX2e

\usepackage{amsmath}
\usepackage{amssymb}
\usepackage{bm}
\usepackage{clrscode3e}
\usepackage{esvect}
\usepackage[T1]{fontenc}
\usepackage{hyperref}
\usepackage{mathtools}
\usepackage{times}
\usepackage{url}

\title{Efficient Calculation of Polynomial Features on Sparse Matrices}


%\author{Andrew Nystrom \\
%Savvysherpa Inc.\\
%6200 Shingle Creek Pkwy \\
%Suite 400 \\
%Minneapolis, MN 55430, USA \\
%\texttt{awnystrom@gmail.com} \\
%\And
%John F. Hughes \\
%Department of Computer Science \\
%Brown University \\
%Providence, RI \\
%\texttt{jfh@cs.brown.edu} \\
%}

\author{
  Nystrom, Andrew\\
  \texttt{awnystrom@gmail.com}
  \and
  Hughes, John\\
  \texttt{jfh@cs.brown.edu}
}

% The \author macro works with any number of authors. There are two commands
% used to separate the names and addresses of multiple authors: \And and \AND.
%
% Using \And between authors leaves it to \LaTeX{} to determine where to break
% the lines. Using \AND forces a linebreak at that point. So, if \LaTeX{}
% puts 3 of 4 authors names on the first line, and the last on the second
% line, try using \AND instead of \And before the third author name.

%\newcommand{\fix}{\marginpar{FIX}}
%\newcommand{\new}{\marginpar{NEW}}
\newcommand{\matr}[1]{\bm{#1}}

\begin{document}


\maketitle

\begin{abstract}
We provide an algorithm for polynomial feature expansion that operates directly on and produces a compressed sparse row matrix without any densification.
For a vector of dimension $D$ and density $d$, the algorithm has time and space complexity $O(d^kD^k)$ where $k$ is the polynomial order, which is an improvement by a factor $d^k$ over the standard method.
\end{abstract}

% this is a test-change for Spike to see whether Git seems to work for him. Ignore. 

\section{Introduction}

Polynomial feature expansion has long been used in statistics to approximate nonlinear functions~\cite{gergonne1974application, smith1918standard}.
The compressed sparse row (CSR) matrix format is a widely-used data structure to hold design matrices for statistics and machine learning applications.
However, polynomial expansions cannot be performed directly on sparse CSR matrices, or any sparse matrix format for that matter, without intermediate densification steps.
This densification not only adds extra overhead, but causes combinations of features that have a product of zero to be computed, then put back into a sparse format, which is futile.

We provide an algorithm that allows CSR matrices to be the input of a polynomial feature expansion without any densification.
The algorithm leverages the CSR format to only compute products of features that result in a nonzero value.
This exploits the sparsity of the data to achieve an improved time complexity of $O(d^kD^k)$ on each vector of the matrix where $k$ is the degree of the expansion, $D$ is the dimensionality, and $d$ is the density.
The standard algorithm has time complexity $O(D^k)$.
Since $0 \le d \le 1$, our algorithm is a significant improvement.
While the algorithm we lay out is uses CSR matrices, it could be slightly modified to operate on other sparse formats.

\section{Preliminaries}
A matrix is uppercase and bold, like this: $\bm{A}$.
In this the context of this paper, row vectors are discussed, but not collum vectors.
Therefore, the distinction between the two is unimportant.
A row vector $\bm{a}_i$ is the $i^{th}$ row of $\bm{A}$ and $\bm{a}$, without a subscript, is a vector not necessarily related to $\bm{A}$.

A compressed sparse row (CSR) matrix representation of $\bm{A}$ consists of three vectors: $\bm{c}$, $\bm{d}$, and $\bm{p}$.
$\bm{c}$ and $\bm{d}$ hold the column indices and data values, respectively, of all nonzero elements of $\bm{A}$.
The values of $\bm{p}$ index $\bm{c}$ and $\bm{d}$ and is itself indexed by rows of $\bm{A}$ such that the nonzero columns of $\bm{a}_i$ are $\bm{c}_{\bm{p}_i:\bm{p}_{i+1}}$ and the nonzero data elements are $\bm{d}_{\bm{p}_i:\bm{p}_{i+1}}$.

\section{Motivation}
In this section, an algorithm for computing polynomial feature expansions on dense matrices will be given.
Next, the algorithm will be modified slightly to operate on a CSR matrix in order to expose its infeasibility.
We will show how the algorithm would be feasible with an added component and derive that component in the following section.

\subsection{Dense Expansion Algorithm}
A natural way to calculate polynomial features for a matrix $\bm{A}$ is to walk down its rows and, for each row, take products of all $k$ combinations of elements.
In order to determine in which column $\bm{A}^k_i$ products of elements in $\bm{A}_i$ belong, a simple counter can be set to zero for each row of $bm{A}$ and incremented efter each polynomial feature is generated.
This counter gives the column of $bm{A}^k_i$ into which each expansion feature belongs.

\begin{codebox}
\footnotesize
\Procname{$\proc{Second Order Dense Polynomial Expansion Algorithm}(\bm{A})$}
    \li $N \gets$ row count of $\bm{A}$
    \li $D \gets$ column count of $\bm{A}$
    \li $\bm{A}^k$ $\gets$ empty $N \times \binom{D}{2}$ matrix
    \li \For $i \gets 0 \To N-1$ \Do
    \li     $c_p \gets 0$
    \li     \For $j_1 \gets 0 \To D-1$ \Do
    \li         \For $j_2 \gets j_1 \To D-1$ \Do
    \li             $\bm{A}^k_{i{c_p}} \gets \bm{A}_{ij_1} \cdot \bm{A}_{ij_2}$
    \li             $c_p \gets c_p + 1$
                \End
            \End
       	\End
\end{codebox}

\subsection{Imperfect CSR Expansion Algorithm}
\label{sec:final-algo}
Now consider how this algorithm might be modified to accept a CSR matrix.
Instead of walking directly down rows of $bm{A}$, we will walk down sections of $\bm{c}$ and $\bm{d}$ partitioned by $\bm{p}$, and instead of inserting polynomial features into $\bm{A}^k$, we will insert column numbers into $\bm{c}^k$ and data elements into $\bm{d}^k$.

\begin{codebox}
\footnotesize
\Procname{$\proc{Incomplete Second Order CSR Polynomial Expansion Algorithm}(\bm{A})$}
    \li $N \gets$ row count of $\bm{A}$
    \li $\bm{p}^k$ $\gets$ vector of size $N+1$
    \li $\bm{p}^k_0 \gets 0$
    \li $nnz^k \gets 0$
    \li \For $i \gets 0 \To N-1$ \Do
    \li     $i_{start} \gets \bm{p}_i$
    \li     $i_{stop} \gets \bm{p}_{i+1}$
    \li     $\bm{c}_i \gets \bm{c}_{i_{start}:i_{stop}}$
    \li     $nnz^k_i \gets \binom{|\bm{c}_i|}{2}$
    \li     $nnz^k \gets nnz^k + nnz^k_i$
    \li     $\bm{p}^k_{i+1} \gets \bm{p}^k_i + nnz^k_i$
        \End
    \zi     
    \zi \Comment Build up the elements of $\bm{p}^k$, $\bm{c}^k$, and $\bm{d}^k$
    \li $\bm{p}^k$ $\gets$ vector of size $N+1$
    \li $\bm{c}^k$ $\gets$ vector of size $nnz^k$
    \li $\bm{d}^k$ $\gets$ vector of size $nnz^k$
    \li $n \gets 0$
    \li \For $i \gets 0 \To N-1$ \Do
    \li     $i_{start} \gets \bm{p}_i$
    \li     $i_{stop} \gets \bm{p}_{i+1}$
    \li     $\bm{c}_i \gets \bm{c}_{i_{start}:i_{stop}}$
    \li     $\bm{d}_i \gets \bm{d}_{i_{start}:i_{stop}}$
    \li     \For $c_1 \gets 0 \To |\bm{c}_i|-1$ \Do
    \li         \For $c_2 \gets c_1 \To |\bm{c}_i|-1$ \Do
%    \li             $j_0 \gets \bm{c}_{c_0}$
%    \li             $j_1 \gets \bm{c}_{c_1}$
    \li             $\bm{d}^k_{n} \gets \bm{d}_{c_0} \cdot \bm{d}_{c_1}$
    \li             $\bm{c}^k_{n} = ?$ \label{li:set_ck}
    \li             $n \gets n + 1$
                \End
            \End
       	\End
\end{codebox}

Note that in the algorithm adapted for CSR matrices, the counter is not reset for each row.
This is because there is only one $\bm{c}^k$ and $\bm{d}^k$ vector for all of $\bm{A}^k$, not one for each row.

However, note line \ref{li:set_ck}.
Given the arbitrary columns involved in a polynomial feature of $\bm{A}_i$, there is no way to determine what the corresponding column of $\bm{A}^k_i$.
In order to achieve a general algorithm, we require a mapping from columns of $\bm{A}$ a column of $\bm{A}^k$.
If there are $D$ columns of $\bm{A}$ and $\binom{D}{k}$ columns of $\bm{A}^k$, this can be accomplished by a bijective mapping of the following form:

\begin{equation}
(j_0, j_1, \dots, j_{k-1}) \rightarrowtail \hspace{-1.9ex} \twoheadrightarrow p_{j_0j_1 \dots i_{k-1}} \in \{0,1,\dots,\binom{D}{k}\} 
\end{equation}

such that $ 0 \le j_0 \le j_1 \le \dots \le j_{k-1} < D$
where $(j_0, j_1, \dots, j_{k-1})$ are elements of $\bm{c}$ and $p_{j_0j_1 \dots i_{k-1}}$ is an element of $\bm{c}^k$. %column indicies of a row vector $\vec{x}$ of an $N \times D$ input matrix, and $p_{i_0i_1 \dots i_{k-1}}$ is a column index into the polynomial expansion vector for $\vec{x}$ where the product of elements corresponding to indices $i_0, i_1, \dots, i_{k-1}$ will be stored.

\section{Construction of Mapping}
For the second degree case, we seek a map from matrix indices $(i, j)$ (with $0 \le i < j < D$ ) to numbers $f(i, j)$ with $0 \le f(i, j) < \frac{D(D-1)}{2}$, one that follows the pattern indicated by 
\begin{align}
\begin{bmatrix}
x & 0 & 1 & 3 \\
x & x & 2 & 4 \\
x & x & x & 5 \\
x & x & x & x
\end{bmatrix}
\label{eq:4x4mat}
\end{align}
where the entry in row $i$, column $j$, displays the value $f(i, j)$. We let $T_2(n) = \frac{1}{2} n(n+1)$ 
be the $n$th triangular number; then in Equation~\ref{eq:4x4mat}, column $j$ (for $j > 0$) contains entries with  
$T_2(j-1) \le e < T_2(j)$; the entry in the $i$th row is just $i + T_2(j-1)$. Thus we have
$
f(i, j) 
= i + T_2(j-1) =  \frac{1}{2}(2i + j^2-j).$
For instance, in column $j = 2$ in our example (the \emph{third} column), the entry in row $i = 1$ is 
$i + T_2(j-1) = 1 + 1 = 2$. 

With one-based indexing in both the domain and codomain, the formula above becomes
$f_1(i, j)  = \frac{1}{2}(2i + j^2 - 3j + 2).$

For \emph{polynomial} features, we seek a similar map $g$, one that also handles the case $i = j$. In this case, a similar analysis yields
$ g(i, j) = i + T_2(j) = \frac{1}{2} (2i + j^2 + j + 1).$


To handle \emph{three-way interactions}, we need to map triples of indices in a 3-index array to a flat list, and similarly for higher-order interactions. For this, we'll need the tetrahedral numbers $T_3(n) = \sum_{i=1}^n T_{2}(n) = 
\frac{1}{6}(n^3 + 3n^2 + 2n)$.

For three indices, $i,j,k$, with $0 \le i < j < k < D$, we have a similar recurrence. Calling the mapping $h$, we have 
\begin{align}
h(i,j,k) = i + T_2(j-1) + T_3(k-2);
\end{align}
if we define $T_1(i) = i$, then this has the very regular form
\begin{align}
h(i,j,k) =  T_1(i) + T_2(j-1) + T_3(k-2);
\end{align}
and from this the generalization to higher dimensions is straightforward. The formulas for ``higher triangular numbers'', i.e., those defined by
\begin{align}
T_k(n) &= \sum_{i=1}^n T_{k-1}(n)
\end{align}
for $k > 1$ can be determined inductively.

The explicit formula for 3-way interactions, with zero-based indexing, is 
\begin{align}
h(i, j, k) &= 1 + (i-1) + \frac{(j-1)j}{2} + \\
& \frac{(k-2)^3 + 3(k-2)^2 + 2(k-2)}{6}. 
\end{align}

\subsubsection{Polynomial Features}
For polynommial features, we seek a map from matrix indices $(i, j)$ (with $i \le j$ and $0 \le i < D$) to numbers $g(i, j)$ with $0 \le f(i, j) < \frac{D(D+1)}{2}$, one that follows the pattern indicated by 
\begin{align}
\begin{bmatrix}
 0 & 1 & 3 & 6 \\
 x & 2 & 4 & 7\\
 x & x & 5 & 8 \\
 x & x & x & 9
\end{bmatrix}
\label{eq:4x4mat-poly}
\end{align}
i.e., essentially the same task as before, except that the diagonal is included. One can regard all but the last column of entries in Equation~\ref{eq:4x4mat-poly} as corresponding to the entries in Equation~\ref{eq:4x4mat}, but shifted to the left. Thus the formula for $g(i, j)$ is simply the formula for $f$, shifted by 1, i.e., 
\begin{align}
g(i, j) &= f(i, j+1)  \\
&=  \frac{2i + (j+1)^2-(j+1)}{2}\\
&=  \frac{2i + j^2 + j + 1)}{2}.
\end{align}
Alternatively, we can write this as
\begin{align}
g(i, j) &= i + T_2(j),
\end{align}
\noindent
and get the same result. 


\subsubsection{Higher dimensions}
To handle three-way interactions, we need to map triples of indices in a 3-index array to a flat list, and similarly for higher-order interactions. 

For three indices, $i,j,k$, with $i < j < k$ and $0 \le i,j,k < D$, we have a similar recurrence. Calling the mapping $h$, we have 
\begin{align}
h(i,j,k) = i + T_2(j-1) + T_3(k-2);
\end{align}
if we define $T_1(i) = i$, then this has the very regular form
\begin{align}
h(i,j,k) =  T_1(i) + T_2(j-1) + T_3(k-2);
\end{align}
and from this, the generalization to higher dimensions is straightforward. The formulas for ``higher triangular numbers'', i.e., those defined by
\begin{align}
T_k(n) &= \sum_{i=1}^n T_{k-1}(n)
\end{align}
for $k > 1$ can be determined inductively. For $k = 3$, the result is 
\begin{align}
T_3(n) &= \sum_{i=1}^n T_{2}(n)\\
&= \frac{n^3 + 3n^2 + 2n}{6},
\end{align}
so that the formula for 3-way interactions, with zero-based indexing, becomes 
\begin{align}
h(i, j, k) &= 1 + (i-1) + \frac{(j-1)j}{2} + \\
& \frac{(k-2)^3 + 3(k-2)^2 + 2(k-2)}{6}. 
\end{align}
\subsubsection{Higher-dimension polynomial features}
For the case where we include the diagonal in higher dimensions, we must shift $j$ by $1$, $k$ by $2$, and so on, and the formula becomes
\begin{align}
\ell(i,j,k) =  T_1(i) + T_2(j) + T_3(k),
\end{align}
with analogous formulas for higher degree polynomial interactions. 

\subsection{Inversion}
We not only want to be able to compute the ``flat index'' $f(i,j)$ corresponding to a particular index pair $(i, j)$, we'd like, given a value $N = f(i, j)$, to be able to determine $i$ and $j$. We know, given $N$, that $f(0, j) \le N \le f(j,j-1)$, i.e., that 
\begin{align}
\frac{j^2 - 3j + 2}{2} &\le N \le \frac{2(j-1) + j^2 - 3j + 2}{2}\\
{j^2 - 3j + 2} &\le 2N \le {j^2 - j}.
\end{align}
If we find the (real number) value $j_{*}$ for which $j_{*}^2 - j_{*} = 2N$, then we know that $j = \lfloor j_{*}\rfloor$. Finding $j_{*}$ amounts to solving a quadratic, whence
\begin{align}
j_{*} = \frac{1 \pm \sqrt{1 + 8N}}{2}.
\end{align}
Clearly the larger value, 
\begin{align}
j_{*} = \frac{1 + \sqrt{1 + 8N}}{2},
\end{align}
is the one we seek; we set $j = \lfloor j_{*}\rfloor$, and compute $i = N - \frac{j^2 - j}{2}$. 
In practice, repeatedly computing square roots may be annoying; instead, when we compute $N = f(i,j)$, we can record the values of $i$ and $j$ along with the interaction features, making inversion a simple matter of looking up these values. 


Similarly, for polynomial features, to invert 
\begin{align}
g(i, j) &=  \frac{2i + j^2 + j + 1)}{2}
\end{align}
we know that if $N = g(i,j)$, then $g(0,j) \le N \le g(j,j)$, which tells us that 
\begin{align}
2N \le j^2 + 3j + 1, \text{ so}\\
0 \le j^2 + 3j + (1-N^2)
\end{align}. 
Solving for $j$ when this last inequality is an equality, we get 
\begin{align}
j_{*} = \frac{-3 \pm \sqrt{ 5 - N^2 }}{2}; 
\end{align}
Once again, we set $j = \lfloor j_{*}\rfloor$, and compute $i = N - \frac{j^2 + j + 1}{2}$. 

Inverting the higher-order formulas, which involve cubics, quartics, etc., is infeasible, but storing the inverse map as described above may be practical, especially for sparse feature sets. 

\section{Final CSR Expansion Algorithm}
With the mapping from columns of $\bm{A}$ to a column of $\bm{A}^k$, we can now write the final form of the innermost loop of the algorithm from \ref{sec:final-algo}.
Let the mapping for $k=2$ be denoted $PolyCol_2$.
Then the innermost loop becomes:

\begin{codebox}
\footnotesize
    \zi         \For $c_2 \gets c_1 \To |\bm{c}_i|-1$ \Do
    \zi             $j_0 \gets \bm{c}_{c_0}$
    \zi             $j_1 \gets \bm{c}_{c_1}$
    \zi             $c_p \gets PolyCol_2(j_0, j_1)$
    \zi             $\bm{d}^k_{n} \gets \bm{d}_{c_0} \cdot \bm{d}_{c_1}$
    \zi             $\bm{c}^k_{n} = c_p$
    \zi             $n \gets n + 1$
                \End
\end{codebox}

\section{Time Complexity}
\subsection{Analytical}
Calculating $k$-degree polynomial features via our method for a row $\bm{a}_i$ of a matrix of dimensionality $D$ and density $d$ requires $\binom{dD}{k}$ (with repetition) products, assuming the nonzero elements of $\bm{A}$ are uniformly distributed.
The complexity of the algorithm, for fixed $k \ll dD$, is therefore
\begin{align}
O\left(\binom{dD+k-1}{k}\right) & = O\left(\frac{(dD+k-1)!}{k!(dD-1)!}\right)\\
& = O\left(\frac{(dD+k-1)(dD+k-2) \dots (dD)}{k!}\right)\\
& = O\left((dD+k-1)(dD+k-2) \dots (dD)\right) \mbox{ for } k \ll dD\\
& = O\left(d^kD^k\right)
\end{align}

\subsection{Empirical}
\section{Conclusion}


\bibliography{sparse_poly}
\bibliographystyle{iclr2016_workshop}

\end{document}
