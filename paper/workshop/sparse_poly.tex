\documentclass{article} % For LaTeX2e
\usepackage{iclr2016_workshop,times}
\usepackage{hyperref}
\usepackage{url}

\usepackage{amsmath}
\usepackage{mathtools}
\usepackage{amssymb}


\title{Efficient Calculation of Polynomial Features on Sparse Matrices}


\author{Andrew Nystrom \\
Savvysherpa Inc.\\
6200 Shingle Creek Pkwy \\
Suite 400 \\
Minneapolis, MN 55430, USA \\
\texttt{awnystrom@gmail.com} \\
\And
John Hughes \\
Department of Computer Science \\
Brown University \\
Providence, RI \\
\texttt{jfh@cs.brown.edu} \\
}

% The \author macro works with any number of authors. There are two commands
% used to separate the names and addresses of multiple authors: \And and \AND.
%
% Using \And between authors leaves it to \LaTeX{} to determine where to break
% the lines. Using \AND forces a linebreak at that point. So, if \LaTeX{}
% puts 3 of 4 authors names on the first line, and the last on the second
% line, try using \AND instead of \And before the third author name.

\newcommand{\fix}{\marginpar{FIX}}
\newcommand{\new}{\marginpar{NEW}}

\begin{document}


\maketitle

\begin{abstract}
We provide an algorithm for polynomial feature expansion that can operate directly on sparse a
matrix. For a vector $\vec{x}$ of a $N \times D$ matrix, the algorithm has time and space complexity $O(d^kD^k)$ where $k$ is the polynomial order and $d$ is the density of $\vec{x}$.

\end{abstract}

\section{Introduction}

Polynomial feature expansion is a tool long used in statistics for approximating nonlinear functions (see \cite{gergonne1974application, smith1918standard}).
While their use is widespread, the authors of this work are unaware of any improvements made to their calculation efficiency.
In this work we provide an algorithm for calculating polynomial features for a row vector $\vec{x}$ of a $N \times D$ matrix with time and space complexity $O(d^kD^k)$ where $k$ is the polynomial order and $d$ is the density of $\vec{x}$.
The density of a vector is the percent of elements that are nonzero, so $0 \le d \le 1$.
The standard algorithm has time and space complexity $O(D^kN)$, so the added factor of $d^k$ represents a significant reduction in time.
The algorithm does not require the densification of the matrix, e.i. the matrix remains in sparse form, so the space complexity is also $O(d^kD^k)$ as opposed to $O(D^k)$.

\section{Algorithm}
The traditional method of calculating polynomial features for a vector $\vec{x}$ involves augmenting a feature for the product of each combination of features in $\vec{x}$ of orders 2 to $k$.
This method does not exploit the sparsity of a sparse matrix and will yield a product of zero any time one of the features involved in the product is zero.
In a sparse matrix, products resulting in zero will be common.
Since the default value of a sparse matrix is zero, these products are entirely unnecessary to compute.

The main idea behind our algorithm is to only compute products that do not involve zeros.
In a compressed sparse row matrix, the columns containing nonzero data are the only columns that are stored.
We can therefore iterate over products of combinations of orders 2 to $k$ of these columns for each row to calculate $k$-degree polynomial features.

While the idea is straightforward, there is yet one unaddressed caveat.
Given a set of columns whose corresponding nonzero components were multiplied to produce a polynomial feature, where in the augmented polynomial vector does the result of the product belong?
To adress this, we give a bijective mapping from the set of possible column index combinations of order 2 to $k$ onto the column index space of the polynomial feature matrix.
More precisely, this the mapping is of the following form:

\begin{equation}
(x_{i_0}, x_{i_1}, \dots, x_{i_{k-1}}) \rightarrowtail \hspace{-1.9ex} \twoheadrightarrow p_{i_0i_1 \dots i_{k-1}} \in \{0,1,\dots, \binom{D}{k}\} 
\forall (i_0, i_1, \dots, i_{k-1})
\end{equation}
such that $ 0 \le i_0 \le i_1 \le \dots \le i_{k-1} < D$
where $(x_{i_0}, x_{i_1}, \dots, x_{i_{k-1}})$ are column indicies of a row vector $\vec{x}$ of an $N \times D$ input matrix and $p_{i_0i_1 \dots i_{k-1}}$ is a column index into the polynomial expansion vector for $\vec{x}$.

\subsection{Construction of Mappings}

We seek a map from matrix indices $(i, j)$ (with $i < j$ and $0 \le i < D$) to numbers $f(i, j)$ with $0 \le f(i, j) < \frac{D(D-1)}{2}$, one that follows the pattern indicated by 
\begin{align}
\begin{bmatrix}
x & 0 & 1 & 3 \\
x & x & 2 & 4 \\
x & x & x & 5 \\
x & x & x & x
\end{bmatrix}
\label{eq:4x4mat}
\end{align}
where the entry in row $i$, column $j$, displays the value $f(i, j)$. 

To simplify slightly, we introduce a notation for the $n$th triangular number, 
\begin{align}
T_2(n) = \frac{n(n+1)}{2}
\end{align}
\noindent
The subscript $2$ indicates that these are triangles in two dimensions; we'll use $T_3(n)$ to indicate the $n$th tetrahedral number, and so on for higher dimensions. 

Observe that in Equation~\ref{eq:4x4mat}, each entry in column $j$ (for $j > 0$) lies in the range
\begin{align}
T_2(j-1) &\le e < T_2(j).
\end{align}
\noindent
And in fact, the entry in the $i$th row of that column is just $i + T_2(j-1)$. Thus we have
\begin{align}
f(i, j) 
&= i + T_2(j-1)\\
&= i + \frac{(j-1)j}{2}\\
&=  \frac{2i + j^2-j}{2}.
\end{align}

For instance, in column $j = 2$ in our example (the \emph{third} column), the entries range from $1$ to $2$, while $T_2(j-1) = T_2(1) = 1$ and $T_2(j) = T_2(2) = 3$, and the entry in column $j = 2$, row $i = 1$ is 
$i + T_2(j-1) = 1 + 1 = 2$. 

\subsubsection{Other indices}
With one-based indexing in both the domain and codomain, the formula above becomes
\begin{align}
f_1(i, j) &= 1+ f(i-1, j-1) \\
&= 1+ f(i-1, j-1) \\
& = \frac{2 + 2(i-1) + (j-1)^2-(j-1)}{2}\\
& = \frac{2i + j^2 - 3j + 2}{2}
\end{align}

\subsubsection{Polynomial Features}
For polynommial features, we seek a map from matrix indices $(i, j)$ (with $i \le j$ and $0 \le i < D$) to numbers $g(i, j)$ with $0 \le f(i, j) < \frac{D(D+1)}{2}$, one that follows the pattern indicated by 
\begin{align}
\begin{bmatrix}
 0 & 1 & 3 & 6 \\
 x & 2 & 4 & 7\\
 x & x & 5 & 8 \\
 x & x & x & 9
\end{bmatrix}
\label{eq:4x4mat-poly}
\end{align}
i.e., essentially the same task as before, except that the diagonal is included. One can regard all but the last column of entries in Equation~\ref{eq:4x4mat-poly} as corresponding to the entries in Equation~\ref{eq:4x4mat}, but shifted to the left. Thus the formula for $g(i, j)$ is simply the formula for $f$, shifted by 1, i.e., 
\begin{align}
g(i, j) &= f(i, j+1)  \\
&=  \frac{2i + (j+1)^2-(j+1)}{2}\\
&=  \frac{2i + j^2 + j + 1)}{2}.
\end{align}
Alternatively, we can write this as
\begin{align}
g(i, j) &= i + T_2(j),
\end{align}
\noindent
and get the same result. 


\subsubsection{Higher dimensions}
To handle three-way interactions, we need to map triples of indices in a 3-index array to a flat list, and similarly for higher-order interactions. 

For three indices, $i,j,k$, with $i < j < k$ and $0 \le i,j,k < D$, we have a similar recurrence. Calling the mapping $h$, we have 
\begin{align}
h(i,j,k) = i + T_2(j-1) + T_3(k-2);
\end{align}
if we define $T_1(i) = i$, then this has the very regular form
\begin{align}
h(i,j,k) =  T_1(i) + T_2(j-1) + T_3(k-2);
\end{align}
and from this, the generalization to higher dimensions is straightforward. The formulas for ``higher triangular numbers'', i.e., those defined by
\begin{align}
T_k(n) &= \sum_{i=1}^n T_{k-1}(n)
\end{align}
for $k > 1$ can be determined inductively. For $k = 3$, the result is 
\begin{align}
T_3(n) &= \sum_{i=1}^n T_{2}(n)\\
&= \frac{n^3 + 3n^2 + 2n}{6},
\end{align}
so that the formula for 3-way interactions, with zero-based indexing, becomes 
\begin{align}
h(i, j, k) &= 1 + (i-1) + \frac{(j-1)j}{2} + \\
& \frac{(k-2)^3 + 3(k-2)^2 + 2(k-2)}{6}. 
\end{align}
\subsubsection{Higher-dimension polynomial features}
For the case where we include the diagonal in higher dimensions, we must shift $j$ by $1$, $k$ by $2$, and so on, and the formula becomes
\begin{align}
\ell(i,j,k) =  T_1(i) + T_2(j) + T_3(k),
\end{align}
with analogous formulas for higher degree polynomial interactions. 

\section{Analysis}
\subsection{Analytical}
\subsection{Empirical}
\section{Conclusion}


\section{Citations, figures, tables, references}
\label{others}

These instructions apply to everyone, regardless of the formatter being used.

\subsection{Citations within the text}

Citations within the text should be based on the {\tt natbib} package
and include the authors' last names and year (with the ``et~al.'' construct
for more than two authors). When the authors or the publication are
included in the sentence, the citation should not be in parenthesis (as
in ``See \citet{Hinton06} for more information.''). Otherwise, the citation
should be in parenthesis (as in ``Deep learning shows promise to make progress towards AI~\citep{Bengio+chapter2007}.'').

The corresponding references are to be listed in alphabetical order of
authors, in the \textsc{References} section. As to the format of the
references themselves, any style is acceptable as long as it is used
consistently.

\subsection{Footnotes}

Indicate footnotes with a number\footnote{Sample of the first footnote} in the
text. Place the footnotes at the bottom of the page on which they appear.
Precede the footnote with a horizontal rule of 2~inches
(12~picas).\footnote{Sample of the second footnote}

\subsection{Figures}

All artwork must be neat, clean, and legible. Lines should be dark
enough for purposes of reproduction; art work should not be
hand-drawn. The figure number and caption always appear after the
figure. Place one line space before the figure caption, and one line
space after the figure. The figure caption is lower case (except for
first word and proper nouns); figures are numbered consecutively.

Make sure the figure caption does not get separated from the figure.
Leave sufficient space to avoid splitting the figure and figure caption.

You may use color figures.
However, it is best for the
figure captions and the paper body to make sense if the paper is printed
either in black/white or in color.
\begin{figure}[h]
\begin{center}
%\framebox[4.0in]{$\;$}
\fbox{\rule[-.5cm]{0cm}{4cm} \rule[-.5cm]{4cm}{0cm}}
\end{center}
\caption{Sample figure caption.}
\end{figure}

\subsection{Tables}

All tables must be centered, neat, clean and legible. Do not use hand-drawn
tables. The table number and title always appear before the table. See
Table~\ref{sample-table}.

Place one line space before the table title, one line space after the table
title, and one line space after the table. The table title must be lower case
(except for first word and proper nouns); tables are numbered consecutively.

\begin{table}[t]
\caption{Sample table title}
\label{sample-table}
\begin{center}
\begin{tabular}{ll}
\multicolumn{1}{c}{\bf PART}  &\multicolumn{1}{c}{\bf DESCRIPTION}
\\ \hline \\
Dendrite         &Input terminal \\
Axon             &Output terminal \\
Soma             &Cell body (contains cell nucleus) \\
\end{tabular}
\end{center}
\end{table}

\section{Final instructions}
Do not change any aspects of the formatting parameters in the style files.
In particular, do not modify the width or length of the rectangle the text
should fit into, and do not change font sizes (except perhaps in the
\textsc{References} section; see below). Please note that pages should be
numbered.

\section{Preparing PostScript or PDF files}

Please prepare PostScript or PDF files with paper size ``US Letter'', and
not, for example, ``A4''. The -t
letter option on dvips will produce US Letter files.

Consider directly generating PDF files using \verb+pdflatex+
(especially if you are a MiKTeX user).
PDF figures must be substituted for EPS figures, however.

Otherwise, please generate your PostScript and PDF files with the following commands:
\begin{verbatim}
dvips mypaper.dvi -t letter -Ppdf -G0 -o mypaper.ps
ps2pdf mypaper.ps mypaper.pdf
\end{verbatim}

\subsection{Margins in LaTeX}

Most of the margin problems come from figures positioned by hand using
\verb+\special+ or other commands. We suggest using the command
\verb+\includegraphics+
from the graphicx package. Always specify the figure width as a multiple of
the line width as in the example below using .eps graphics
\begin{verbatim}
   \usepackage[dvips]{graphicx} ...
   \includegraphics[width=0.8\linewidth]{myfile.eps}
\end{verbatim}
or % Apr 2009 addition
\begin{verbatim}
   \usepackage[pdftex]{graphicx} ...
   \includegraphics[width=0.8\linewidth]{myfile.pdf}
\end{verbatim}
for .pdf graphics.
See section 4.4 in the graphics bundle documentation (\url{http://www.ctan.org/tex-archive/macros/latex/required/graphics/grfguide.ps})

A number of width problems arise when LaTeX cannot properly hyphenate a
line. Please give LaTeX hyphenation hints using the \verb+\-+ command.


\subsubsection*{Acknowledgments}

Use unnumbered third level headings for the acknowledgments. All
acknowledgments, including those to funding agencies, go at the end of the paper.

\bibliography{sparse_poly}
\bibliographystyle{iclr2016_workshop}

\end{document}
