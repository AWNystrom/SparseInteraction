\documentclass[11pt]{article}

\usepackage{array}
\usepackage{clrscode3e}
\usepackage{amsmath}

\setlength{\parindent}{0em}
\setlength{\parskip}{1em}

\begin{document}

\title{Efficient Calculation of Interaction Features on Sparse Matrices}
\author{Andrew Nystrom}
\date{}

\maketitle

\begin{abstract}%   <- trailing '%' for backward compatibility of .sty file
FILL THIS IN
\end{abstract}

\section{Introduction}

Introduction
Interaction features are a way of capturing correlations between features in a machine 
learning setting. They consist of products of combinations of features. This work 
describes a method for efficiently calculating second degree interaction features on a 
sparse matrix, and could be generalized to higher orders.

Consider the following matrix:

\[ \left( \begin{array}{cccc}
3 & 0 & 0 & 3 \\
0 & 4 & 2 & 0 \\
1 & 2 & 0 & 3 \end{array} \right)\]


If each row is an instance vector, then the interaction feature matrix is

\[ \left( \begin{array}{cccccc}
0 & 0 & 9 & 0 & 0 & 0 \\
0 & 0 & 0 & 8 & 0 & 0 \\
2 & 0 & 3 & 0 & 6 & 0 \end{array} \right)\]

Note that there are $\binom{D}{2} = \frac{D^2-D}{2}$ columns in the interaction matrix, which is D choose 2, 
since we generate products of all combinations of 2 features in the original matrix. 
Which column corresponds with which product pair is not important so long as it’s 
consistent. In this example, each column in the interaction matrix corresponded to the 
following product pairs: $(1, 2),  (1, 3),  (1, 4),  (2, 3), ( 2, 4), (3, 4)$.

Notice that the interaction matrix contains many zero entries. This is of course because 
the original matrix contained zero entries, so the interaction features, which are 
products of pairs of features, contain many zeros. This means that the only products that 
need to actually be calculated or those for which both features in the combination are 
nonzero. If the original matrix is sparse and represented in a sparse matrix format 
(e.g. compressed sparse row), a list of nonzero column indices are stored for each row 
and are easily retrievable in $O(1)$ time. Interaction features can be generated from this 
list via the following method.

\section{Approach}
Let the list of nonzero columns for a given row be $N_{zc}$. The nonzero second degree 
interaction features are simply the products of all combinations of two elements whose 
columns are in $N_{zc}$. However, to do this efficiently, a consistent mapping from the column 
index pairs of $N_{zc}$ into the columns of the interaction matrix is needed. The mapping is 
from the space (a, b) where a,b are in $1,2,..., D$ onto the space $1,2,..., \frac{D^2-D}{2}$. This 
is isomorphic to mapping the coordinates of the upper triangle of a matrix onto a flat 
list. The following is a proof by construction for such a mapping.

INSERT JOHN'S PROOF HERE

With this mapping, an algorithm for generating second degree interaction features on a 
matrix $A$ can be formulated as follows:

\begin{codebox}
\Procname{$\proc{Sparse Interaction}(A)$}
    \zi $\func{map}(b, a) = \frac{2Da-a^2+2b-3a-2}{2}$
    \zi $B$ $\gets$ Compressed Sparse Row Matrix of size $N \times \frac{D^2-D}{2}$
    \zi \For $\id{row}$ in $A$ \Repeat
    \zi     $N_{zc} \gets$ nonzero columns of $row$
    \zi     \For each combination $a, b$ of elements of $N_{zc}$ \Repeat
    \zi         $k \gets \func{map}(b, a)$
    \zi         $i \gets$ index of $\id{row}$
    \zi         $B[i, k] \gets \id{row}[a] \cdot \id{row}[b]$
            \End
       	\End
\end{codebox}

\section{Complexity Analysis}
Assume that A is a matrix with sparsity $0 < d < 1$, $N$ rows, and $D$ columns. Finding 
interaction features with the proposed algorithm has time and space complexity 
$\func{O}(d N D^2)$, 
whereas a naive approach of using non-sparse matrices and multiplying all column 
combinations has time and space complexity $\func{O}(N D^2)$. The algorithm is therefore an 
improvement by a factor of the density factor of $A$.

This can represent a large gain in speed and time. For example, the 20 Newsgroups dataset 
has density $d$ of 0.12 when its unigrams are represented in a vector space model. This 
means the proposed approach would take less than $\frac{1}{8}$ time and memory.

The real benefit of this method is revealed when the average complexity is analysed. The 
number of interactions calculated for a given row are $\binom{|N_{zc}|}{2}$. If the matrix has 
density $d$, then on average, $N_{zc} = D d$, so the number of interaction features 
calculated in total is 

\begin{align*}
N \binom{d D}{2} &= \frac{N (Dd)!}{2!(Dd-2)!}\\
    \\
    &= N \frac{(D^2d^2-Dd)}{2}
\end{align*}

This means that the average complexity decreases quadratically with the density.

\section{Future Work}
The approach for generating second degree interaction features required a mapping from 
combinations of two to the space $1,2,...,\frac{D^2-D}{2}$, which is isomorphic to a mapping from 
the indices of an upper triangular matrix to the indices of a flat list of the same size. 
To generate third degree interaction features, a mapping from combinations of three 
$(a,b,c)$ to the space $1,2,...\frac{D^3-3D^2+2D}{6}$ (which is $\binom{D}{3}$), or the upper $3$-simplex of a tensor to a flat 
list of the same size $\frac{D^3-3D^2+2D}{6}$ would be required. In general, for interaction 
features of degree k, the upper $k$-simplex of a $k$-dimensional tensor must be mapped to the 
space $1,2,...\frac{D!}{k!(D-k)!}$. A similar approach for finding these mappings could be taken 
as the one used here for $k=2$. 

Motivation for deriving mapping functions for higher orders
of interaction features is that the average complexity of generating degree $k$ interaction
features is $N \binom{Dd}{k}$, which decreases polynomially with respect to k compared to
generating the features naively.

    
\vskip 0.2in
\bibliography{sample}

\end{document}